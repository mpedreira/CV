%%%%%%%%%%%%%%%%%%%%%%%%%%%%%%%%%%%%%%%%%
% Twenty Seconds Resume/CV
% LaTeX Template
% Version 1.0 (14/7/16)
%
% Original author:
% Carmine Spagnuolo (cspagnuolo@unisa.it) with major modifications by 
% Vel (vel@LaTeXTemplates.com) and Harsh (harsh.gadgil@gmail.com)
%
% License:
% The MIT License (see included LICENSE file)
%
%%%%%%%%%%%%%%%%%%%%%%%%%%%%%%%%%%%%%%%%%

%----------------------------------------------------------------------------------------
%	PACKAGES AND OTHER DOCUMENT CONFIGURATIONS
%----------------------------------------------------------------------------------------
\documentclass[a4paper]{twentysecondcv} % a4paper for A4
\usepackage[none]{hyphenat}

% Command for printing skill overview bubbles
\newcommand\skills{ 
~
	\smartdiagram[bubble diagram]{
        \textbf{Manuel}\\\textbf{Pedreira},
        \textbf{Web}\\\textbf{Performance},
        \textbf{Web}\\\textbf{Security},
        \textbf{Engineering},
        \textbf{Modern}\\\textbf{Design},
        \textbf{Analytical}\\\textbf{Skill},
        \textbf{Ecommerce}
    }
}

% Programming skill bars
\languajes{{Spanish   / 6}, {Galician  / 6}, {English $\textbullet$ CAE-C1/ 4.5}, {Portuguese / 3.5}}


%----------------------------------------------------------------------------------------
%	 PERSONAL INFORMATION
%----------------------------------------------------------------------------------------
% If you don't need one or more of the below, just remove the content leaving the command, e.g. \cvnumberphone{}

\cvname{Manuel Pedreira} % Your name
\cvjobtitle{ Enterprise Architect } % Job
% title/career

\cvlinkedin{/in/manuelpa}
\cvgithub{mpedreira}
\cvnumberphone{(+34) 647722535} % Phone number
%\cvsite{} 
\cvmail{manuelpa@gmail.com} % Email address

%----------------------------------------------------------------------------------------
% Projects text
\certifications{
%\textbf{AKAMAI Aqua Ion}{AKAMAI} \\
%\textbf{AKAMAI Bot Manager Foundations}{AKAMAI}\\
%\textbf{AKAMAI Dynamic Site Acceleration}{AKAMAI}\\ 
%\textbf{AKAMAI Kona Site Defender - Configuration and Maintenance}{AKAMAI} \\
%\textbf{AKAMAI Web Peformance Foundations}{AKAMAI} \\
%\textbf{AKAMAI Web Performance Offload}{AKAMAI} \\
%\textbf{AZ 900 - Microsoft Azure Fundamentals}{Microsoft} \\
%\textbf{Fortinet FCNSA}{FORTINET} \\
%\textbf{Fortinet FCNSP}{FORTINET} \\
\textbf{ITIL Foundation} - {EXIM} \\
\textbf{Professional Scrum Product Owner I} - {SCRUM.net} \\ 
\textbf{SCRUM Master} - {SCRUM Manager}\\
}
\aboutme{
IT professional with solid technical education and more than 20 years of experience  that is capable of understanding the the business challenges and design solutions and execute aligned with it.
I am accustomed to manage and coordinate teams of different sizes including extremely big teams for running technical projects in environments where the change and the exigence are in the day basis} %About me section

\begin{document}

\makeprofile % Print the sidebar
%\makebox[0pt][l]{%
%  \raisebox{-\totalheight}[0pt][0pt]{%
%    \includegraphics[width=0.3in]{img/quote.png}}}%{\raggedright\color{dark-gray}
%\quote{IT professional with solid technical education and more than 20 years of experience  that is capable of %understanding the the business challenges and design solutions and execute aligned with it.
%I am accustomed to manage and coordinate teams of different sizes including extremely big teams for running technical %projects in environments where the change and the exigence are in the day basis}  

%----------------------------------------------------------------------------------------
%	 EXPERIENCE
%----------------------------------------------------------------------------------------https://es.overleaf.com/project/6284a571a95fe235b9c0fb79

\section{Experience}

\begin{twenty} % Environment for a list with descriptions
\twentyitem
    	{May.2021 -}
		{Present}
        {DMO Area Manager (Digital)}
        {\href{https://www.inditex.com/}{\textbf{INDITEX}}}
        {Arteixo}
        {
        I am in charge of Digital Management Office for Digital Department and Zara.com. \\
        As part of my responsibilities are control of the annual budget of the department (+150M), the relationship with the providers, the financial control of Digital Department and the control of the deliveries of the products and services of all the department (+2000 technicians)
        }
        \\
\twentyitem
    	{Feb.2019 -}
		{Present}
        {SMO Area Manager(Digital)}
        {\href{https://www.inditex.com/}{\textbf{INDITEX}}}
        {Arteixo}
        {
        Throughout my time as the Service Management Office Lead, we made, among others, the roll-out of a ITSM platform for all technology operations and support, we defined an score card for operation departments (+1200 users), a gitOps based tool for alerting of production issues used by all technical departments in INDITEX or a BPM with noCode approach that helps support departments to define and execute workarounds in INDITEX eCommerce platforms.
        }
        \\
\twentyitem
    	{Feb.2017 -}
		{Feb.2019}
        {Systems Architecture Team Lead}
        {\href{https://www.inditex.com/}{\textbf{INDITEX}}}
        {Arteixo}
        {
        I was the Systems Architecture Team Lead for INDITEX eCommerce platforms. During this period, I was responsible for the definition an execution of the new architecture of the INDITEX eCommerce platform and the production maintenance of the actual platform (+4000 Servers).\\ 
        During this years, I was the IOP Infrastructure lead architect \footnote{\href {https://www.inditex.com/documents/10279/304402/2022+Horizon_eng.pdf/0c482bbd-692e-1346-5af5-0c6bf1c5e02f}{Horizon 2022}}. This project that has started in 2017 represents more than the 80\% of the total sales of INDITEX eCommerce platforms.This platform, based on microservices approach, became the standard platform for INDITEX transformation projects in 2019.\\
        During this period, I have had more than 100 people working in my team and a budget of more than 25M year
        }
        \\
\twentyitem
    	{Feb.2017 -}
		{Oct.2009}
        {Deputy CTO}
        {\href{https://www.inditex.com/}{\textbf{INDITEX}}}
        {Arteixo}
        {
        My role as part of the Technical Office was to define and control the core projects of IT Department in the scope of eCommerce, System Architecture and operations. During these years, some of  my key projects and responsibilities were:
        \begin{itemize}
            \item \textbf{eCommerce Datacenter Design:} This project had the requirement of defining a solution for INDITEX eCommerce platforms where we could have more than 250k orders/hour (we were able to overcome this objective in more than 1000\%) and an availability of 100\%. In this project that have had a budget the more than 30M, my responsibilities were not only to define and validate an active active solution for datacenters, networking and application servers but also review the execution plan for warranty that the implementation was aligned with the design.\\
            \item \textbf{Infrastructure Lead for eCommerce platforms} including opening Zara.com and other brands and China market\\
            \item \textbf{Lead the chapter of architecture design of INDITEX}        
        \end{itemize}
        }
        \\
\twentyitem
    	{Sep.2009 -}
		{Feb.2009}
        {Security Technical Manager}
        {\href{https://www.inditex.com/}{\textbf{INDITEX}}}
        {Arteixo}
        {
        During my time as Technical Manager in Security Department, I was in charge of selecting and implementing security solution like the IAM solution -in 2020 it was managing more than 250k users- or the corporate SIEM solution. 
        }
        \\  
\end{twenty}
\newpage
\makeprofilepagetwo % Print the sidebar
\begin{twenty}  
\twentyitem
    	{Feb.2009 -}
		{Feb.2008}
        {Systems Architecture Project Manager}
        {\href{https://www.inditex.com/}{\textbf{INDITEX}}}
        {Arteixo}
        {
        My duties as project manager were to warranty the success of key projects in the department like the active directory consolidation, the application deployment process or the TGT \footnote{\href {https://www.inditex.com/documents/10279/245898/Memoria_Anual_2007.pdf/74c34dcb-6502-4078-a13c-a4d50c55676b}{2007 INDITEX Anual Report}} project that was the first interface for online Communications between stores and headquarters 
        }
        \\
\twentyitem
    	{Feb.2008 -}
		{Feb.2006}
        {Systems Architect}
        {\href{https://www.inditex.com/}{\textbf{INDITEX}}}
        {Arteixo}
        {
        I was lead of the designing and implementation of some projects from our Area like the email platform migration to Exchange 2003 or the implementation of the first eCommerce platform in INDITEX that was Zara Home in 2007\\
        Apart of these projects, I was in charge of the design and implementation of some branch office projects like the active directory migration in Portugal and France or the integration of the INDITEX Turkish office
        }
\twentyitem
    	{Feb.2006 -}
		{Feb.2004}
        {System Administrator and Support}
        {\href{https://www.inditex.com/}{\textbf{INDITEX}}}
        {Arteixo}
        {
        During this stage I was in charge of the maintenance of Windows NT directory, storage, firewalls and email corporate platforms.\newline{}
        I was also in charge of the support of more than 7000 users in headquarters and branch offices
        }
\twentyitem
    	{Feb.2004 -}
		{May.2003}
        {System Administrator and Support}
        {\href{http://www.nogueirayalvarez.es/}{\textbf{Gestor\'ia Nogueira y \'Alvarez}}}
        {Ourense}
        {
        I was in charge of the support and administrator of the office. During this year, I was in charge of the Windows NT to Windows 2003 migration and the administration of the website and email service.
        }
	%\twentyitem{<dates>}{<title>}{<location>}{<description>}
\end{twenty}

%----------------------------------------------------------------------------------------
%	 EDUCATION
%----------------------------------------------------------------------------------------
\section{Education}

\begin{twenty} % Environment for a list with descriptions
	\twentyitem
    	{2015}{}{Executive Development Programme}{\href{https://www.ie.edu/es/}{\textbf{Instituto de Empresa(IE)}}}{}{}
	\twentyitem
    	{2011 - 2013}{}{Executive Master of Business Administration(MBA)}{\href{https://www.nebrija.com/}{\textbf{Univ. Nebrija}}}{}{}
	\twentyitem
    	{2004 - 2006}{}{Software Libre International Master}{\href{https://www.uoc.edu/}{\textbf{Universitat Oberta de Catalunya}}}{}{}
	\twentyitem
    	{2002 - 2004}{}{Software Engineering}{\href{https://esei.uvigo.es/}{\textbf{Universidade de Vigo}}}{}{}
	\twentyitem
    	{1998 - 2002}{}{Management Software Engineering}{\href{https://esei.uvigo.es/}{\textbf{Universidade de Vigo}}}{}{}
	%\twentyitem{<dates>}{<title>}{<organization>}{<location>}{<description>}
\end{twenty}

\section{Courses}

\begin{twenty} % Environment for a list with descriptions
    \twentyitem{2016}{}{Direcci\'on y Gesti\'on de Proyectos TI bajo la Metodolog\'ia PMI } {Vitae Consultores}{}{}
    \twentyitem{2016}{}{Data Scentists's Toolbox } {Coursera}{}{}
%    \twentyitem{2014}{}{Fortinet FCNSA Preparation Exam } {FORTINET}{}{}
%    \twentyitem{2014}{}{Fortinet FCNSP Preparation Exam} {FORTINET}{}{}
%    \twentyitem{2014}{}{AKAMAI KONA} {AKAMAI}{}{}
    \twentyitem{2013}{}{Oratoria ApreHender a transmitir} {Bevel}{}{}
%    \twentyitem{2013}{}{IBM WebSphere Commerce V7 System Administration} {IBM formaci\'on}{}{}
%    \twentyitem{2012}{}{The ITIL Foundation v3.0} {Xunta - CNTG}{}{}
%    \twentyitem{2011}{}{RSA enVision Advanced Administration} {RSA}{}{}
%    \twentyitem{2011}{}{AKAMAI Dynamic Site Acceleration} {AKAMAI}{}{}
%    \twentyitem{2011}{}{Programa formativo en Habilidades: Eficacia Personal - Excelencia Profesional} {INDRA}{}{}
%    \twentyitem{2011}{}{Programa formativo en Habilidades} {INDRA}{}{}
%    \twentyitem{2010}{}{Fundamentos e bases de Cobit} {Xunta - CNTG}{}{}
%    \twentyitem{2016}{}{The ITIL Foundation Certification v3.0}{Element-K}{}{}
%    \twentyitem{2016}{}{Introducci\'on a la metodolog\'ia de direcci\'on de proyectos} {Tecnocom}{}{}
%    \twentyitem{2010}{}{Formaci\'on de OCS 2007} {Profesional Training}{}{}
%    \twentyitem{2010}{}{MOC5178A Implementing and Maintaining Audio/Visual Conferencing and Web Conferencing Using OCS 2007} {Profesional Training}{}{}
%    \twentyitem{2010}{}{MOC5179A Implementing and Maintaining Telephony Using Microsoft Office Communications Server 2007} {Profesional Training}{}{}
%    \twentyitem{2010}{}{Obradoiro sobre Xesti\'on de Identidades e Control de Accesos } {Oracle}{}{}
%    \twentyitem{2010}{}{Formaci\'on DB2 especial para cliente R7624ES} {IBM formaci\'on}{}{}
%    \twentyitem{2010}{}{Using TSA with InfoSphere Balanced Warehouse} {IBM formaci\'on}{}{}
%    \twentyitem{2008}{}{Administraci\'on e Instalaci\'on Peoplenet} {Meta4}{}{}
%    \twentyitem{2008}{}{MOC6425 Configuring Windows Server 2008 Active Directory Domain Services} {Profesional Training}{}{}
%    \twentyitem{2008}{}{MOC6436 Designing a Windows Server 2008 Active Directory Infraestructure and Services} {Profesional Training}{}{}
%    \twentyitem{2008}{}{MOC5053 Designing a Messaging Infraestructure using Microsoft Exchange Server 2007 } {Profesional Training}{}{}
%    \twentyitem{2008}{}{MOC5054 Designing a High Availability Messaging Solution with Microsoft Exchange Server2007} {Profesional Training}{}{}
%    \twentyitem{2008}{}{Curso de GNU/Linux (medio)} {Profesional Training}{}{}
%    \twentyitem{2008}{}{Curso de Radware Appdirector} {Radware}{}{}
%    \twentyitem{2007}{}{AIX 5l Administraci\'on del Sistema III: Gesti\'on del Rendimiento} {IBM formaci\'on}{}{}
%    \twentyitem{2007}{}{AIX 5l Administraci\'on del Sistema II} {IBM formaci\'on}{}{}
%    \twentyitem{2007}{}{AIX 5l Administraci\'on del Sistema II} {IBM formaci\'on}{}{}
%    \twentyitem{2007}{}{WorkShop Recovery Exchang} {Microsoft}{}{}
%    \twentyitem{2003}{}{Xornadas de Seguridad Inform\'atica} {Deputación de Ourense}{}{}
	%\twentyitem{<dates>}{<title>}{<organization>}{<location>}{<description>}
\end{twenty}

%----------------------------------------------------------------------------------------
%	 Additional Achievements
%----------------------------------------------------------------------------------------
\section{Other}

\begin{twenty} % Environment for a list with descriptions
	\twentyitem
    	{Present -  }
        {2018}
        {Web Performance Client Advisory Board Member}
        {\href{http://www.akamai.com}{\textbf{AKAMAI}}}
        {}
        {}
        \\
        \\
	%\twentyitem{<dates>}{<title>}{<organization>}{<location>}{<description>}
\end{twenty}
\flushright{ \tiny{Last Update: \today}

\end{document} 
